% This is lnicst.tex the demonstration file of the LaTeX macro package for
% Lecture Notes of the Institute for Computer Sciences, Social-Informatics 
% and Telecommunications Engineering series from Springer-Verlag.
% It serves as a template for authors as well.
% version 1.0 for LaTeX2e
%
\documentclass[lnicst,a4paper]{svmultln}
%
\usepackage{graphicx}
%
\usepackage{makeidx}  % allows for indexgeneration
% \makeindex          % be prepared for an author index
%
\begin{document}
%
\mainmatter              % start of the contribution
%
\title{Humanoid robot obstacle recognition via data filtering, localisation and robot-to-robot communication in context of RoboCup} 
%
\titlerunning{Hinderniserkennung}  % abbreviated title (for running head)
%                                     also used for the TOC unless
%                                     \toctitle is used
%
\author{Benjamin Scholz \and Daniel Speck \and Judith Hartfill}
%
\authorrunning{Benjamin Scholz \and Daniel Speck \and Judith Hartfill}   % abbreviated author list (for running head)
%
%%%% list of authors for the TOC (use if author list has to be modified)
\tocauthor{}
%
\institute{University of Hamburg, Fachbereich Informatik\\
\email{2scholz@informatik.uni-hamburg.de, 2speck@informatik.uni-hamburg.de, 2hartfil@informatik.uni-hamburg.de}
}

\maketitle              % typeset the title of the contribution
% \index{Ekeland, Ivar} % entries for the author index
% \index{Temam, Roger}  % of the whole volume
% \index{Dean, Jeffrey}

\begin{abstract}        % give a summary of your paper
The abstract should summarize the contents of the paper. It will be set in 9-point
font size and be inset 1.0 cm from the right and left margins.
There will be two blank lines before and after the Abstract.

Use the abstract section to provide a teaser for the contents of your report I Do not attempt to write a review or summary I Be concise: Your abstract should have 200 words or less (do
not use more than 250 words)
%                         please supply keywords within your abstract
\keywords {obstacle recognition, obstacle avoidance, localisation, filtering, data smoothing, vision, swarm intelligence}
\end{abstract}
%
\section{Introduction}
 
 
In this project our aim was to improve the way the robots are playing football together. Until then, each robot had its own view/opinion of the surrounding field and so was able to make decisions only due to this information. Our idea was to build an intern team communication system, that provides one robots information for every team member. Thus a form of swarm intelligence could increase the quality of the robots   acting together as a whole team. Considering that, it was necessary to develop a world model to make the information transfomable, unique and easy to provide. Furthermore we detected, that the image prosessing quality was not good enough to precisly determine the robots position and so we did some vision modifications and added some filters. 


Outline of underlying concepts

Brief summary of relevant theoretical background knowledge

Review of existing (published) work relevant for your topic(s)

Motivate the reader for the issue(s) you are trying to solve

Explain why your work (your approach) is necessary
\subsection{Motivation}
Obstacle recognition and dynamis behaviour towards this is very important if several robots shall act together in a certain way. One of the main reasons for this is to avoid the physical contact between robots. As long as their hardware is not good enough to hold balance when beeing touched, they are very likely to fall and - in worst-case-scenario - bring about anothers robots downfall. This usually causes hight costs for hardware fixes and also disturbs the gameplay. Getting up again can take several seconds and toward the opposite robots it is not within the meaning of fairplay.
Besides more and especially better knowledge about the surroundings is strategically important in the RoboCup competition. Having a reliably prospect of its own positions, the position of the ball and the opposites robots position combined with a good path finding algorithm can raise the number of scores significantly.
was war das problem bisher?
\subsection{Problem}
\subsubsection{Vision}
When we checked how good the already implemented algorithm worked we discovered, that ot even the goals were recognized suffinciently. The kind of jumoed from one point to the other and most of the time their were more than two goal postst recognized fo one goal. In addition obstacles at least could be recognise, but there was no behaviour to react in a senseful way. Moreover it was no possible for the robot to distinguish between obstacles, such as team mate or opposite player.

\subsubsection{Filtering}

was haben wir uns als aufgabenstellung gesetzt? (teilprobleme)
\section{Solution}
What did you do and how did you do it?

Methods

Design

Implementation

Do not include every possible detail and avoid redundancy

Use subsections to emphasize certain aspects/components of
your work -
but do not overuse them!

Avoid the passive voice: Y was done by X, use the active voice: X did Y
\subsection{Vision}
\subsection{Filtering}
\subsection{Localisation}
\subsection{Communication}
\section{Results}
Present your results in a logical sequence

Highlight the importance of your results and explain your
analysis methodology

Discuss the results you infer from your work

Important:
Adopt a critical approach in your discussion

Do not oversell your results - put the advantages first, but
don’t forget to mention the shortcomings!
\section{Summary}

Be more informative than your abstract!

Include a concise version of your discussion

Highlight what you found out

Highlight the problems you encountered

Explain how your results support your conclusions!

Provide suggestions for future research and briefly outline how
suggested research can be attempted

Important:
Make this section readable

\section{References}
Very important section of your report

If you used external information/results
)
Provide a
reference!

References will help the reader understand the basis of your
work and provide context for comparison

Use of references might also help you to be more concise

There are several types of reference

Book

Journal article

Conference publication

Web site

Web sites are usually unchecked sources -
be careful


\paragraph{Notes and Comments.}
The first results on subharmonics were
obtained by Foster and Kesselman in \cite{fos:kes}, who showed the existence of
infinitely many subharmonics both in the subquadratic and superquadratic
case, with suitable growth conditions on $H'$. Again the duality
approach enabled Foster and Waterman in \cite{fos:kes:2} to treat the
same problem in the convex-subquadratic case, with growth conditions on
$H$ only.

Recently, Smith and Waterman (see \cite{smit:wat} and May et al. \cite{mes})
have obtained lower bound on the number of subharmonics of period $kT$,
based on symmetry considerations and on pinching estimates, as in
Sect.~5.2 of this article.

%
% ---- Bibliography ----
%
\begin{thebibliography}{5}

\bibitem{smit:wat} Smith, T.F., Waterman, M.S.: Identification of Common Molecular
Subsequences. J. Mol. Biol. 147, 195--197 (1981)

\bibitem{mes} May, P., Ehrlich, H.C., Steinke, T.: ZIB Structure Prediction Pipeline:
Composing a Complex Biological Workflow through Web Services. In: Nagel,
W.E., Walter, W.V., Lehner, W. (eds.) Euro-Par 2006. LNCS, vol. 4128,
pp. 1148--1158. Springer, Heidelberg (2006)

\bibitem{fos:kes} Foster, I., Kesselman, C.: The Grid: Blueprint for a New Computing
Infrastructure. Morgan Kaufmann, San Francisco (1999)

\bibitem{cff} Czajkowski, K., Fitzgerald, S., Foster, I., Kesselman, C.: Grid
Information Services for Distributed Resource Sharing. In: 10th IEEE
International Symposium on High Performance Distributed Computing, pp.
181--184. IEEE Press, New York (2001)

\bibitem{fos:kes:2} Foster, I., Kesselman, C., Nick, J., Tuecke, S.: The Physiology of the
Grid: an Open Grid Services Architecture for Distributed Systems
Integration. Technical report, Global Grid Forum (2002)

\bibitem{url} National Center for Biotechnology Information, http://www.ncbi.nlm.nih.gov

\end{thebibliography}
%
\end{document}
